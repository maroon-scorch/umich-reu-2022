\documentclass{article}
\usepackage[utf8]{inputenc}
\usepackage[english]{babel}

\usepackage{geometry}
\geometry{letterpaper,margin=1in}
\usepackage{graphicx}
\usepackage{enumitem}
\usepackage{amssymb}
\usepackage{amsmath}
\usepackage[all]{xy}
\usepackage{hyperref}
\usepackage{mathabx}
\usepackage{tikz}
\usepackage{mathabx}
\usepackage[]{amsthm} %lets us use \begin{proof}
\DeclareGraphicsRule{.tif}{pnf}{.png}{'convert #1 'dirname #1'/'basename #1.tif'.png}

\newcommand{\bigzero}{\mbox{\normalfont\Large\bfseries 0}}

\usepackage{subfig}

\usepackage{rotating}
\usepackage{tabularx}
\usepackage{caption}
\usepackage{xspace}

\newcommand{\Rbb}{\mathbb{R}}
\newcommand{\Pbb}{\mathbb{P}}


\setlength\parindent{0pt}

\usepackage{listings}
\usepackage{xcolor}

\definecolor{codegreen}{rgb}{0,0.6,0}
\definecolor{codegray}{rgb}{0.5,0.5,0.5}
\definecolor{codepurple}{rgb}{0.58,0,0.82}
\definecolor{backcolour}{rgb}{0.95,0.95,0.92}

\lstdefinestyle{mystyle}{
    backgroundcolor=\color{backcolour},   
    commentstyle=\color{codegreen},
    keywordstyle=\color{magenta},
    numberstyle=\tiny\color{codegray},
    stringstyle=\color{codepurple},
    basicstyle=\ttfamily\footnotesize,
    breakatwhitespace=false,         
    breaklines=true,                 
    captionpos=b,                    
    keepspaces=true,                 
    numbers=left,                    
    numbersep=5pt,                  
    showspaces=false,                
    showstringspaces=false,
    showtabs=false,                  
    tabsize=2
}

\lstset{style=mystyle}


\begin{document}
\theoremstyle{definition}
\newtheorem{theorem}{Theorem}[section]
\theoremstyle{definition}
\newtheorem{conjecture}[theorem]{Conjecture}
\theoremstyle{definition}
\newtheorem{definition}[theorem]{Definition}
\theoremstyle{definition}
\newtheorem{goal}[theorem]{Goal}
\theoremstyle{definition}
\newtheorem{corollary}[theorem]{Corollary}
\theoremstyle{definition}
\newtheorem{question}[theorem]{Question}
\theoremstyle{definition}
\newtheorem{lemma}[theorem]{Lemma}
\theoremstyle{definition}
\newtheorem{proposition}[theorem]{Proposition}

\begin{theorem}
The following are equivalent:
\begin{itemize}
    \item $\pi_2(\Rbb): Y(\Rbb) \to \Pbb^{2}_{[u, v, w]}(\Rbb)$ is surjective
    \item For all $[u, v, w] \in \Pbb^2_{[u, v, w]}(\Rbb)$, the correspondent quadratic form
    \[z^2 = Q_1(u, v, w)t_0^2 + 2Q_2(u, v, w)t_0 t_1 + Q_3(u, v, w) t_1^2\]
    has a real solution
    \item For all $[u, v, w] \in \Pbb^2_{[u, v, w]}(\Rbb)$, the matrix
    \[\begin{pmatrix}
    Q_1(u, v, w) & Q_2(u, v, w) & 0\\ 
    Q_2(u, v, w) & Q_3(u, v, w) & 0\\
    0 & 0 & -1
    \end{pmatrix}\]
    is not negative-definite
    \item For all $[u, v, w] \in \Pbb^2_{[u, v, w]}(\Rbb)$, $Q_1(u, v, w) \geq 0$ or $\Delta(u, v, w) \leq 0$
\end{itemize}
If there exist some $[u_0, v_0, w_0] \in \Pbb^2_{[u, v, w]}(\Rbb)$ such that $Q_1(u, v, w) < 0$ and $\Delta(u, v, w) > 0$, then $\pi_2(\Rbb)$ is not surjective on $[u_0, v_0, w_0]$ 
\end{theorem}

\begin{proposition}
$\pi_1(\Rbb): Y(\Rbb) \to \Pbb^{1}_{[t_0, t_1]}(\Rbb)$ does not imply $\pi_2(\Rbb): Y(\Rbb) \to \Pbb^{2}_{[u, v, w]}(\Rbb)$ is surjective.
\end{proposition}

\begin{proof}
Consider the following Quadratic Forms:
\begin{itemize}
    \item $Q1 = -1*(3/11*u^2 + 14*u*v + 8/11*v^2 + 24/13*u*w + 2/13*v*w + 5/7*w^2)$
    \item $Q2 = 4/7*u^2 + 6*u*v + 2/3*v^2 + 12/13*u*w + 6*v*w + 3*w^2$
    \item $Q3 = -1*(2*u^2 + 4/7*u*v + 1/7*v^2 + 11*u*w + 24/11*v*w + 3/11*w^2)$
\end{itemize}
We first note that the induced $\pi_2(\Rbb)$ is not surjective onto $[1, 0, 0]$ since by Theorem 0.1
\[Q_1(1, 0, 0) = -\frac{3}{11} < 0\]
\[\Delta(1, 0, 0) = (-\frac{3}{11})(-2) - (\frac{4}{7})^2 = \frac{6}{11} - \frac{16}{49} = \frac{118}{539} > 0\]
But computing the numerical signature for $\pi_1(\Rbb)$ shows that $\pi_1(\Rbb)$ is surjective onto $\Pbb^{1}_{[t_0, t_1]}(\Rbb)$:
\begin{lstlisting}[language=Matlab, caption=$\pi_2(\mathbb{R})$ is Not Surjective]
Roots:  
[ <-0.947412429179734260860628575406, 1>,
<-0.0668965206372537373058194027354, 1>,
<0.144493936058127884925785447134, 1>,
<0.799785515079943374910685955052, 1>,
<2.16394992543614689928080382193, 1>,
<5.88640499958028118487838169057, 1> ]
------------------------
Evaluated at: ( -0.507154474908493999083223989071 , 1)
2 2
Evaluated at: ( 0.0387987077104370738099830221994 , 1)
1 3
Evaluated at: ( 0.472139725569035629918235701093 , 1)
2 2
Evaluated at: ( 1.48186772025804513709574488849 , 1)
1 3
Evaluated at: ( 4.02517746250821404207959275624 , 1)
2 2
Evaluated at: ( 6.88640499958028118487838169057 , 1)
1 3
\end{lstlisting}
\end{proof}

\begin{proposition}
$\pi_2(\Rbb): Y(\Rbb) \to \Pbb^{2}_{[u, v, w]}(\Rbb)$ does not imply $\pi_1(\Rbb): Y(\Rbb) \to \Pbb^{1}_{[t_0, t_1]}(\Rbb)$ is surjective.
\end{proposition}

\begin{proof}
Consider the following Quadratic Forms:
\begin{itemize}
    \item $Q_1 = u^2 + v^2 + w^2$
    \item $Q_2 = -99/34*u^2 - 51/77*v^2 - 19/22*w^2$
    \item $Q3 = 70/17*u^2 - 16/77*v^2 + 6/11*w^2$
\end{itemize}
Since $Q_1$ is positive definite, $\pi_2(\Rbb)$ is surjective onto $\Pbb^{2}_{[u, v, w]}(\Rbb)$. However, explicitly computing the numerical signature for $\pi_1(\Rbb)$ shows that:
\begin{lstlisting}[language=Matlab, caption=$\pi_1(\mathbb{R})$ is Not Surjective]
Roots:
[ <-0.141704243748784752857711368579, 1>,
<0.415961009009267967193354026598, 1>,
<0.823529411764705882352941176432, 1>,
<1.31131171826345930553391870091, 1>,
<1.46637956842410942818238669305, 1>,
<5.00000000000000000000000000001, 1> ]
------------------------
Evaluated at: ( 0.137128382630241607167821329010 , 1)
2 2
Evaluated at: ( 0.619745210386986924773147601515 , 1)
1 3
Evaluated at: ( 1.06742056501408259394342993867 , 1)
0 4
Evaluated at: ( 1.38884564334378436685815269698 , 1)
1 3
Evaluated at: ( 3.23318978421205471409119334653 , 1)
2 2
Evaluated at: ( 6.00000000000000000000000000001 , 1)
3 1
\end{lstlisting}
So $\pi_1(\Rbb)$ is not surjective.
\end{proof}

\end{document}
