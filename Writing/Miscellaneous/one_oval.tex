\documentclass{article}
\usepackage[utf8]{inputenc}
\usepackage[english]{babel}

\usepackage{geometry}
\geometry{letterpaper,margin=1in}
\usepackage{graphicx}
\usepackage{enumitem}
\usepackage{amssymb}
\usepackage{amsmath}
\usepackage[all]{xy}
\usepackage{hyperref}
\usepackage{mathabx}
\usepackage{tikz}
\usepackage{mathabx}
\usepackage[]{amsthm} %lets us use \begin{proof}
\DeclareGraphicsRule{.tif}{pnf}{.png}{'convert #1 'dirname #1'/'basename #1.tif'.png}

\newcommand{\bigzero}{\mbox{\normalfont\Large\bfseries 0}}

\usepackage{subfig}

\usepackage{rotating}
\usepackage{tabularx}
\usepackage{caption}
\usepackage{xspace}

\newcommand{\Rbb}{\mathbb{R}}
\newcommand{\Pbb}{\mathbb{P}}


\setlength\parindent{0pt}

\begin{document}
\theoremstyle{definition}
\newtheorem{theorem}{Theorem}[section]
\theoremstyle{definition}
\newtheorem{conjecture}[theorem]{Conjecture}
\theoremstyle{definition}
\newtheorem{definition}[theorem]{Definition}
\theoremstyle{definition}
\newtheorem{goal}[theorem]{Goal}
\theoremstyle{definition}
\newtheorem{corollary}[theorem]{Corollary}
\theoremstyle{definition}
\newtheorem{question}[theorem]{Question}
\theoremstyle{definition}
\newtheorem{lemma}[theorem]{Lemma}
\theoremstyle{definition}
\newtheorem{proposition}[theorem]{Proposition}

\begin{lemma}
Let $Y_{\Tilde{\Delta}/\Delta}$ be determined by
\[z^2 = Q_1(u, v, w)t_0^2 + 2Q_2(u, v, w)t_0 t_1 + Q_3(u, v, w) t_1^2\]
Let $X_{\Tilde{\Delta}/\Delta}$ be determined by
\[z^2 = Q_3(u, v, w)t_0^2 + 2Q_2(u, v, w)t_0 t_1 + Q_1(u, v, w) t_1^2\]
Then $Y(\Rbb)$ is connected if and only if $X(\Rbb)$ is connected.
\end{lemma}

\begin{proof}
The map $f: Y(\Rbb) \to X(\Rbb)$ that swaps $t_0$ and $t_1$ is a homeomorphism.
\end{proof}

\begin{lemma}
$\pi_2(\Rbb)(Y(\Rbb)) = \{p \in \Pbb^2_{[u:v:w]}(\Rbb)\ |\ Q_1(p) \geq 0 \text{ or } \Delta(p) \leq 0\}$
\end{lemma}

\begin{proposition}
If the topological type of $\Delta(\Rbb)$ is one oval, then $Y(\Rbb)$ is connected.
\end{proposition}

\begin{proof}
It suffices for us to show that $\pi_2(\Rbb)(Y(\Rbb))$ is connected, since every fiber of the image is connected and $\pi_2$ is a close map.\\\\
If $Q_1$ is positive definite, then $\pi_2(\Rbb)(Y(\Rbb)) = \Pbb^2_{[u:v:w]}(\Rbb)$ is connected.\\\\
If $Q_1$ is negative definite, then $\pi_2(\Rbb)(Y(\Rbb)) = \{p \in \Pbb^2_{[u:v:w]}(\Rbb)\ | \Delta(p) \leq 0\}$, which is connected since $\Delta(\Rbb)$ is one oval, so both $\Delta(\Rbb)$ and its complement is connected.\\\\
If $Q_1$ is indefinite, assume for the sake of contradiction that $Y(\Rbb)$ is disconnected, since $(Q_1 \geq 0)$ and $(\Delta(p) \leq 0)$ are both connected, it has to be the case that $\pi_2(Y(\Rbb))$ has 2 connected components being $(Q_1 \geq 0)$ and $(\Delta(p) \leq 0)$.\\\\
Now consider the set $(Q_3 \geq 0)$. If $Q_3$ is negative-definite or positive definite, then we are done by just substituting $Q_1$ with $Q_3$ and $Q_3$ with $Q_1$ using Lemma 0.1. Now if $Q_3$ is indefinite, then $(Q_3 \geq 0)$ is a connected non-empty set.\\\\
Let $S_1 = (Q_1 \geq 0)$ and $S_2 = (\Delta(p) \leq 0)$, we note that for all $p \in \Pbb^2(\Rbb) \setminus (S_1 \cup S_2)$, $\Delta(p) > 0$ and $Q_1(p) \leq 0$. So in particular $Q_1(p)Q_3(p) = \Delta(p) + Q_2(p)^2 > 0$ implies that $Q_3(p) < 0$.\\\\
Now for points in $S_1$ and $S_2$, we note that $Q_3$ cannot have both $p \in S_1$ and $q \in S_2$ such that $Q_3(p) \geq 0$ and $Q_3(q) \geq 0$, because this would imply that $(Q_3 \geq 0)$ has two connected components. Thus, $(Q_3 \geq 0)$ can only be contained in one of $S_1$ and $S_2$.\\\\
If $(Q_3 \geq 0)$ is contained in $S_2$, then we can again make the substitution of $Q_1$ with $Q_3$ and $Q_3$ and $Q_1$ using Lemma 0.1.\\\\
If $(Q_3 \geq 0)$ is contained in $S_1$, we note that for all point $p$ in $(Q_1 > 0)$, $Q_1(p) > 0$ and $\Delta(p) = Q_1(p)Q_3(p) - Q_2^2(p) > 0$, so $Q_1(p)Q_3(p) > Q_2^2(p) \geq 0$ implies that $Q_3(p) > 0$. Therefore, we know that for all $p \in (Q_1 > 0)$, $Q_3(p) > 0$, and for all $p \in (Q_1 < 0)$, $Q_3(p) < 0$.\\\\
But since $Q_3$ is smooth, there has to be some point $q \in (Q_1 = 0)$ such that $Q_3(q) = 0$, but this means that at that point. $Q_1(q) = Q_3(q) = 0$, so $\Delta(q) = 0 - Q_2(q)^2 \leq 0$, so $q \in (Q_1 \geq 0) \cap (\Delta \leq 0)$, which is a contradiction.\\\\
Thus, we conclude that $Y(\Rbb)$ is connected.
\end{proof}

\end{document}
