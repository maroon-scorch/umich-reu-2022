\documentclass{article}
\usepackage[utf8]{inputenc}
\usepackage[english]{babel}

\usepackage{geometry}
\geometry{letterpaper,margin=1in}
\usepackage{graphicx}
\usepackage{enumitem}
\usepackage{amssymb}
\usepackage{amsmath}
\usepackage[all]{xy}
\usepackage{hyperref}
\usepackage{mathabx}
\usepackage{tikz}
\usepackage{mathabx}
\usepackage[]{amsthm} %lets us use \begin{proof}
\DeclareGraphicsRule{.tif}{pnf}{.png}{'convert #1 'dirname #1'/'basename #1.tif'.png}

\newcommand{\bigzero}{\mbox{\normalfont\Large\bfseries 0}}

\usepackage{subfig}

\usepackage{rotating}
\usepackage{tabularx}
\usepackage{caption}
\usepackage{xspace}

\newcommand{\Rbb}{\mathbb{R}}
\newcommand{\Pbb}{\mathbb{P}}


\setlength\parindent{0pt}

\begin{document}
\theoremstyle{definition}
\newtheorem{theorem}{Theorem}[section]
\theoremstyle{definition}
\newtheorem{conjecture}[theorem]{Conjecture}
\theoremstyle{definition}
\newtheorem{definition}[theorem]{Definition}
\theoremstyle{definition}
\newtheorem{goal}[theorem]{Goal}
\theoremstyle{definition}
\newtheorem{corollary}[theorem]{Corollary}
\theoremstyle{definition}
\newtheorem{question}[theorem]{Question}
\theoremstyle{definition}
\newtheorem{lemma}[theorem]{Lemma}
\theoremstyle{definition}
\newtheorem{proposition}[theorem]{Proposition}

\begin{theorem}
The following are equivalent:
\begin{itemize}
    \item $\pi_1(\Rbb): Y(\Rbb) \to \Pbb^{1}_{[t_0, t_1]}(\Rbb)$ is surjective
    \item For all $[t_0, t_1] \in \Pbb^{1}_{[t_0, t_1]}(\Rbb)$, the correspondent quadratic form:
    \[z^2 = Q_1(u, v, w)t_0^2 + 2Q_2(u, v, w)t_0 t_1 + Q_3(u, v, w) t_1^2\]
    has a real solution
    \item Let $M_1, M_2, M_3$ be the symmetric matrix associated to $Q_1, Q_2, Q_3$ the matrix 
    \[M_{[t_0, t_1]} = 
\left(\begin{array}{@{}c|c@{}}
  \begin{matrix}
    M_1t_0^2 + 2M_2t_0t_1 + M_3t_1^2
  \end{matrix}
  & \bigzero \\
\hline
  \bigzero &
  \begin{matrix}
  -1
  \end{matrix}
\end{array}\right)
\]
    is indefinite
    \item $M_{[t_0, t_1]}$ is not negative-definite (since the matrix cannot be positive definite with the $-1$ term)
\end{itemize}
\end{theorem}

\begin{theorem}[Sylvester's Criterion]
Let $M \in M_{n \times n}(\Rbb)$ be an $n \times n$ real matrix, and let $M_1, ..., M_n$ be real matrices such that $M_k$ is the $k \times k$ upper left corner matrix of $M$.\\\\
Then $M$ is negative-definite if and only if for all odd $k$, $det(M_k) < 0$, and for all even $k$, $det(M_k) > 0$.
\end{theorem}

\begin{proposition}
Suppose either $M_1$ or $M_3$ is negative definite, then $\pi_1(\Rbb): Y(\Rbb) \to \Pbb^{1}_{[t_0, t_1]}(\Rbb)$ is not surjective.
\end{proposition}

\begin{proof}
Suppose $M_1$ is negative definite, then on $[1, 0] \in \Pbb^{1}_{[t_0, t_1]}(\Rbb)$, the matrix $M_{[t_0, t_1]}$ becomes
\[M_{[1, 0]} = \left(\begin{array}{@{}c|c@{}}
  \begin{matrix}
    M_1
  \end{matrix}
  & \bigzero \\
\hline
  \bigzero &
  \begin{matrix}
  -1
  \end{matrix}
\end{array}\right)\]
Let $M_1, M_2, M_3, M_4$ be the upper left corner matrix as described in Theorem 0.2. Since $M_1$ is negative definite, we have that $det(M_1) < 0, det(M_2) > 0, det(M_3) < 0$. We also have that
\[det(M_4) = det(M) = (-1)det(M_3) > 0\]
So Theorem 0.2 shows that $M_{[1, 0]}$ is a negative definite matrix, then Theorem 0.1 shows that $\pi_1(\Rbb)$ is not surjective.\\\\
Suppose $M_3$ is negative definite, then a nearly identical argument follows by considering the point $[0, 1] \in \Pbb^{1}_{[t_0, t_1]}(\Rbb)$
\end{proof}

\end{document}
