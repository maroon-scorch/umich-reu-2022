\documentclass{article}
\usepackage[utf8]{inputenc}
\usepackage[english]{babel}

\usepackage{geometry}
\geometry{letterpaper,margin=1in}
\usepackage{graphicx}
\usepackage{enumitem}
\usepackage{amssymb}
\usepackage{amsmath}
\usepackage[all]{xy}
\usepackage{hyperref}
\usepackage{mathabx}
\usepackage{tikz}
\usepackage{mathabx}
\usepackage[]{amsthm} %lets us use \begin{proof}
\DeclareGraphicsRule{.tif}{pnf}{.png}{'convert #1 'dirname #1'/'basename #1.tif'.png}

\newcommand{\bigzero}{\mbox{\normalfont\Large\bfseries 0}}

\usepackage{subfig}

\usepackage{rotating}
\usepackage{tabularx}
\usepackage{caption}
\usepackage{xspace}

\newcommand{\Rbb}{\mathbb{R}}
\newcommand{\Pbb}{\mathbb{P}}


\setlength\parindent{0pt}

\begin{document}
\theoremstyle{definition}
\newtheorem{theorem}{Theorem}[section]
\theoremstyle{definition}
\newtheorem{conjecture}[theorem]{Conjecture}
\theoremstyle{definition}
\newtheorem{definition}[theorem]{Definition}
\theoremstyle{definition}
\newtheorem{goal}[theorem]{Goal}
\theoremstyle{definition}
\newtheorem{corollary}[theorem]{Corollary}
\theoremstyle{definition}
\newtheorem{question}[theorem]{Question}
\theoremstyle{definition}
\newtheorem{lemma}[theorem]{Lemma}
\theoremstyle{definition}
\newtheorem{proposition}[theorem]{Proposition}

\begin{lemma}
$\pi_2(\Rbb)(Y(\Rbb)) = \{p \in \Pbb^2_{[u:v:w]}(\Rbb)\ |\ Q_1(p) \geq 0 \text{ or } \Delta(p) \leq 0\}$
\end{lemma}

\begin{proposition}
If the topological type of $\Delta(\Rbb)$ is one oval, then $Y(\Rbb)$ is connected.
\end{proposition}

\begin{proof}
It suffices for us to show that $\pi_2(\Rbb)(Y(\Rbb))$ is connected, since every fiber of the image is connected and $\pi_2$ is a close map.\\\\
If $Q_1$ is positive definite, then $\pi_2(\Rbb)(Y(\Rbb)) = \Pbb^2_{[u:v:w]}(\Rbb)$ is connected.\\\\
If $Q_1$ is negative definite, then $\pi_2(\Rbb)(Y(\Rbb)) = \{p \in \Pbb^2_{[u:v:w]}(\Rbb)\ | \Delta(p) \leq 0\}$, which is connected since $\Delta(\Rbb)$ is one oval, so both $\Delta(\Rbb)$ and its complement is connected.\\\\
If $Q_1$ is indefinite, assume for the sake of contradiction that $Y(\Rbb)$ is disconnected, since $(Q_1 \geq 0)$ and $(\Delta(p) \leq 0)$ are both connected, it has to be the case that $\pi_2(Y(\Rbb))$ has 2 connected components being $(Q_1 \geq 0)$ and $(\Delta(p) \leq 0)$.\\\\
But we note that for all $p \in (Q_1 = 0)$, we have that $\Delta(p) = -Q_2(p) \leq 0$, so $(Q_1 = 0)$ is contained in $(\Delta(p) \leq 0)$. But this means that $\pi_2(Y(\Rbb))$ is clearly connected. So we have a contradiction.\\\\
Thus, in all cases, $Y(\Rbb)$ is connected.
\end{proof}

\end{document}
