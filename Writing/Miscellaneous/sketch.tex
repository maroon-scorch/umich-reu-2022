\documentclass{article}
\usepackage[utf8]{inputenc}
\usepackage[english]{babel}

\usepackage{geometry}
\geometry{letterpaper,margin=1in}
\usepackage{graphicx}
\usepackage{enumitem}
\usepackage{amssymb}
\usepackage{amsmath}
\usepackage[all]{xy}
\usepackage{hyperref}
\usepackage{mathabx}
\usepackage{tikz}
\usepackage{mathabx}
\usepackage[]{amsthm} %lets us use \begin{proof}
\DeclareGraphicsRule{.tif}{pnf}{.png}{'convert #1 'dirname #1'/'basename #1.tif'.png}

\newcommand{\bigzero}{\mbox{\normalfont\Large\bfseries 0}}

\usepackage{subfig}

\usepackage{rotating}
\usepackage{tabularx}
\usepackage{caption}
\usepackage{xspace}

\newcommand{\Rbb}{\mathbb{R}}
\newcommand{\Pbb}{\mathbb{P}}


\setlength\parindent{0pt}

\begin{document}
\theoremstyle{definition}
\newtheorem{theorem}{Theorem}[section]
\theoremstyle{definition}
\newtheorem{conjecture}[theorem]{Conjecture}
\theoremstyle{definition}
\newtheorem{definition}[theorem]{Definition}
\theoremstyle{definition}
\newtheorem{goal}[theorem]{Goal}
\theoremstyle{definition}
\newtheorem{corollary}[theorem]{Corollary}
\theoremstyle{definition}
\newtheorem{question}[theorem]{Question}
\theoremstyle{definition}
\newtheorem{lemma}[theorem]{Lemma}
\theoremstyle{definition}
\newtheorem{proposition}[theorem]{Proposition}

\begin{theorem}
The following are equivalent:
\begin{itemize}
    \item $\pi_1(\Rbb): y(\Rbb) \to \Pbb^{1}_{[t_0, t_1]}(\Rbb)$ is surjective
    \item For all $[t_0, t_1] \in \Pbb^{1}_{[t_0, t_1]}(\Rbb)$, the correspondent quadratic form:
    \[z^2 = Q_1(u, v, w)t_0^2 + 2Q_2(u, v, w)t_0 t_1 + Q_3(u, v, w) t_1^2\]
    has a real solution
    \item Let $M_1, M_2, M_3$ be the symmetric matrix associated to $Q_1, Q_2, Q_3$ the matrix 
    \[M_{[t_0, t_1]} = 
\left(\begin{array}{@{}c|c@{}}
  \begin{matrix}
    M_1t_0^2 + 2M_2t_0t_1 + M_3t_1^2
  \end{matrix}
  & \bigzero \\
\hline
  \bigzero &
  \begin{matrix}
  -1
  \end{matrix}
\end{array}\right)
\]
    is indefinite
    \item $M_{[t_0, t_1]}$ is not negative-definite (since the matrix cannot be positive definite with the $-1$ term)
\end{itemize}
\end{theorem}

\begin{lemma}
If $\Delta(\Rbb) = \O$, then $\Tilde{\Delta}(\Rbb) = \O$
\end{lemma}

\begin{proof}
Suppose for contradiction that $\Tilde{\Delta}(\Rbb) \neq \O$. Then there exist some non-zero $(u, v, w, r, s)$ such that
\[Q_1(u, v, w) = r^2, Q_2(u, v, w) = r*s, Q_3(u, v, w) = s^2\]
but this means that
\[Q_2(u, v, w)^2 - Q_1(u, v, w)Q_3(u, v, w) = (rs)^2 - r^2s^2 = 0\]
has solution $(u, v, w)$.\\\\
Clearly $(u, v, w) \neq (0, 0, 0)$ as if they do, then this would imply that $0 = r^2, 0 = s^2$, which means that $r, s = 0$. Thus, $\Delta(\Rbb)$ then does have some non-zero real solution, so $\Delta(\Rbb) \neq \O$, which is a contradiction.\\\\
\end{proof}

\begin{proposition}
There exist $Y_{\Tilde{\Delta}/\Delta}$ such that $\pi_1(\Rbb)$ is surjective and $\Tilde{\Delta}(\Rbb) = \O$.
\end{proposition}

\begin{proof}
Let $M_1 = [8/7, 3, 2], M_2 = [7/5, 3/5, 10/7], M_3 = [2, 1, 8/5]$, then the matrix becomes
\[M_{[t_0, t_1]} = \begin{bmatrix} \frac{8}{7} t_0^2 + \frac{14}{5} t_0 t_1 + 2 t_1^2 & 0 & 0 & 0\\
0 & 3t_0^2 + \frac{6}{5}t_0t_1 + t_1^2 & 0 & 0\\
0 & 0 & 2t_0^2 + \frac{20}{7}t_0t_1 + \frac{8}{5} t_1^2 & 0\\
0 & 0 & 0 & -1
\end{bmatrix}\]
The code in PVS showed that the $\Delta(\Rbb)$ for this setup is empty, so Lemma 0.2 gives us that $\Tilde{\Delta}(\Rbb) = \O$.\\
It remains for us to show that, for all $[t_0, t_1]$, $M_{[t_0, t_1]}$ is not negative definite. Indeed, Sylvester's Criterion tells us that it suffices for us to show that
$\frac{8}{7} t_0^2 + \frac{14}{5} t_0 t_1 + 2 t_1^2 > 0$ for all non-zero $(t_0, t_1)$, but
\begin{align*}
    \frac{8}{7} t_0^2 + \frac{14}{5} t_0 t_1 + 2 t_1^2 &= \frac{8}{7}(t_0^2 + \frac{14 \cdot 7}{40} t_0 t_1 + \frac{7}{4}t_1^2)\\
    &= \frac{8}{7}(t_0^2 + \frac{49}{20}t_0 t_1 + \frac{7}{4}t_1^2)\\
    &= \frac{8}{7}[(t_0 + \frac{49}{40}t_1)^2 + (\frac{7}{4} - (\frac{49}{40})^2)t_0^2] \tag*{Completing the Square}\\
    &\geq \frac{8}{7}[\frac{7}{4} - (\frac{49}{40})^2](t_0^2)
\end{align*}
Note that $\frac{8}{7}(\frac{7}{4} - (\frac{49}{40})^2 > 0$, so the expression is strictly greater than $0$ when $t_0 \neq 0$, so Theorem 0.1 tells us that $\pi_1(\Rbb)$ is surjective.
\end{proof}

\begin{proposition}
The following two statements are true:
\begin{itemize}
    \item If $Q_1(\Rbb) \cap \Delta(\Rbb) \neq \O$ and if $Q_3$ is positive definite, then $\Tilde{\Delta}$ has an $\Rbb$-point
    \item If there exist $p \in \Pbb^2_{[u, v, w]}(\Rbb)$ such that $\Delta(p) = 0$ and either $Q_1(p) > 0$ or $Q_3(p) > 0$, then $\Tilde{\Delta}$ has an $\Rbb$-point at $p$.
\end{itemize}
\end{proposition}

\begin{proposition}
Under the assumption that $Q_1$ and $\Delta$ have real points ($Q_1$ itself is indefinite), then $Y(\Rbb)$ is disconnected if and only if the following are both true
\begin{itemize}
    \item 1) The real points of $Q_1 = 0$ and $\Delta = 0$ are disjoint
    \item 2) The locus where $Q_1 > 0$ is disjoint from the locus where $\Delta < 0$.
\end{itemize}
\end{proposition}

\begin{proof}
Suppose $Y(\Rbb)$ is disconnected, assume for the sake of contradiction that $(1)$ is false, then there exist some point $p$ such that $\Delta(p) = Q_1(p) = 0$. In particular $\Delta(p) = 0 = Q_1(p)Q_3(p) - Q_2^2(p) = -Q_2^2(p)$, so $Q_2(p) = 0$. Then clearly $Y$ is $\Rbb$-rational by choosing $t_0 = 1, t_1 = 0, z = 0, (u, v, w) = p$, so $Y(\Rbb)$ is connected, hence contradiction.\\\\
Assume now that $(2)$ is false, then there exist some point $p$ such that $Q_1(p) > 0$ and $\Delta(p) < 0$, then the fiber of $p$ has an indefinite matrix, so $Y$ is $\Rbb$-rational and thus connected.
\\\\
Conversely, clearly $Y(\Rbb)$ is surjective on a point if and only if its associated matrix is indefinite. The matrix at point $p$ is indefinite if and only if $Q_1(p) > 0$ or $\Delta(p) < 0$. So the image of $Y(\Rbb)$ is the set $\{p | Q_1(p) > 0 \text{ or } \Delta(p) < 0\}$, which is disjoint by our assumption. Since $\pi_2$ is continuous and the continuous image of a connected space is connected, this implies that $Y(\Rbb)$ is not connected.
\end{proof}

\begin{proposition}
If $Q_1$ is positive definite, then $\pi_2(\Rbb)$ is surjective on $\Pbb^2_{[u, v, w]}(\Rbb)$, which implies that $Y(\Rbb)$ is connected.
\end{proposition}

\begin{proposition}
If the topological type of $\Delta(\Rbb)$ is one oval, then $Y(\Rbb)$ is connected.
\end{proposition}

\begin{proof}
The image $\pi_2(\Rbb)(Y(\Rbb))$ is exactly $\{p \in \Pbb^2_{[u:v:w]}(\Rbb)\ |\ Q_1(p) \geq 0 \text{ or } \Delta(p) \leq 0\}$.\\\\
If $Q_1$ is positive definite, then $\pi_2(\Rbb)(Y(\Rbb)) = \Pbb^2_{[u:v:w]}(\Rbb)$ is connected.\\\\
If $Q_1$ is negative definite, then $\pi_2(\Rbb)(Y(\Rbb)) = \{p \in \Pbb^2_{[u:v:w]}(\Rbb)\ | \Delta(p) \leq 0\}$, which is connected since $\Delta(\Rbb)$ is one oval, so both $\Delta(\Rbb)$ and its complement is connected.\\\\
If $Q_1$ is indefinite, assume for the sake of contradiction that $Y(\Rbb)$ is disconnected, since $(Q_1 \geq 0)$ and $(\Delta(p) \leq 0)$ are both connected, it has to be the case that $\pi_2(Y(\Rbb))$ has 2 connected components being $(Q_1 \geq 0)$ and $(\Delta(p) \leq 0)$.\\\\
Now consider the set $(Q_3 \geq 0)$. If $Q_3$ is negative-definite or positive definite, then we are done by just substituting $Q_1$ with $Q_3$ and $Q_3$ with $Q_1$. Now if $Q_3$ indefinite, then $(Q_3 \geq 0)$ is a connected non-empty set.\\\\
Let $S_1 = (Q_1 \geq 0)$ and $S_2 = (\Delta(p) \leq 0)$, we note that for all $p \in \Pbb^2(\Rbb) \setminus (S_1 \cup S_2)$, $\Delta(p) > 0$ and $Q_1(p) \leq 0$. So in particular $Q_1(p)Q_3(p) = \Delta(p) + Q_2(p)^2 > 0$ implies that $Q_3(p) < 0$.\\\\
Now for points in $S_1$ and $S_2$, we note that $Q_3$ cannot have both $p \in S_1$ and $q \in S_2$ such that $Q_3(p) \geq 0$ and $Q_3(q) \geq 0$, because this would imply that $(Q_3 \geq 0)$ has two connected components. Thus, $(Q_3 \geq 0)$ can only be contained in one of $S_1$ and $S_2$.\\\\
If $(Q_3 \geq 0)$ is contained in $S_2$, then we can again make the substitution of $Q_1$ with $Q_3$ and $Q_3$ and $Q_1$.\\\\
If $(Q_3 \geq 0)$ is contained in $S_1$, we note that for all point $p$ in $(Q_1 > 0)$, $Q_1(p) > 0$ and $\Delta(p) = Q_1(p)Q_3(p) - Q_2^2(p) > 0$, so $Q_1(p)Q_3(p) > Q_2^2(p) \geq 0$ implies that $Q_3(p) > 0$. Therefore, we know that for all $p \in (Q_1 > 0)$, $Q_3(p) > 0$, and for all $p \in (Q_1 < 0)$, $Q_3(p) < 0$.\\\\
But since $Q_3$ is smooth, there has to be some point $q \in (Q_1 = 0)$ such that $Q_3(q) = 0$, but this means that at that point. $Q_1(q) = Q_3(q) = 0$, so $\Delta(q) = 0 - Q_2(q)^2 \leq 0$, so $q \in (Q_1 \geq 0) \cap (\Delta \leq 0)$, which is a contradiction.\\\\
Thus, we conclude that $Y(\Rbb)$ is connected.
\end{proof}

\end{document}
