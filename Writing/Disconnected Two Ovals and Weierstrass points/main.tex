\documentclass{article}
\usepackage[utf8]{inputenc}
\usepackage[english]{babel}

\usepackage{geometry}
\geometry{letterpaper,margin=1in}
\usepackage{graphicx}
\usepackage{enumitem}
\usepackage{amssymb}
\usepackage{amsmath}
\usepackage[all]{xy}
\usepackage{hyperref}
\usepackage{mathabx}
\usepackage{tikz}
\usepackage{mathabx}
\usepackage[]{amsthm} %lets us use \begin{proof}
\DeclareGraphicsRule{.tif}{pnf}{.png}{'convert #1 'dirname #1'/'basename #1.tif'.png}

\newcommand{\bigzero}{\mbox{\normalfont\Large\bfseries 0}}

\usepackage{subfig}

\usepackage{rotating}
\usepackage{tabularx}
\usepackage{caption}
\usepackage{xspace}

\newcommand{\Rbb}{\mathbb{R}}
\newcommand{\Pbb}{\mathbb{P}}

\begin{document}
\theoremstyle{definition}
\newtheorem{theorem}{Theorem}[section]
\theoremstyle{definition}
\newtheorem{conjecture}[theorem]{Conjecture}
\theoremstyle{definition}
\newtheorem{definition}[theorem]{Definition}
\theoremstyle{definition}
\newtheorem{goal}[theorem]{Goal}
\theoremstyle{definition}
\newtheorem{corollary}[theorem]{Corollary}
\theoremstyle{definition}
\newtheorem{question}[theorem]{Question}
\theoremstyle{definition}
\newtheorem{lemma}[theorem]{Lemma}
\theoremstyle{definition}
\newtheorem{proposition}[theorem]{Proposition}

\begin{proposition}
    If $Y_{\Tilde{\Delta}/\Delta}(\Rbb)$ has two connected components and the hyper-elliptic curve $\Gamma$ has six real Weierstrass points, then $\Delta(\Rbb)$ is two nested ovals.
\end{proposition}

\begin{proof}
It suffices for us to show that $\Delta(\Rbb)$ is not two non-nested ovals, so we will assume for the sake of contradiction that it is. Since $\Gamma$ has six real Weierstrass points and $Y_{\Tilde{\Delta}/\Delta}(\Rbb)$ has two connected components, the signatures of $\pi_1(\Rbb)$ must follow $(0, 4), (1, 3), (0, 4), (1, 3), (2, 2), (1, 3)$. In particular, $\pi_1(\Rbb)$ contains some point $t$ whose fiber has signature $(2, 2)$.\\\\
Theorem 2.6(1) of FSJVV then tells us that, we can without loss choose $Q_1$ such that its associated matrix $M_1$ has 1 negative eigenvalue and 2 positive eigenvalues. Geometrically, $(Q_1 \geq 0)$ is the closure of the compliment of a disk.\\\\
Since $Q_1$ is indefinite and $Y_{\Tilde{\Delta}/\Delta}(\Rbb)$  is disconnected, the image of $Y_{\Tilde{\Delta}/\Delta}(\Rbb)$ under $\pi_2(\Rbb)$ has to be $(Q_1 Q_3 - Q_2^2 \leq 0)$, meaning that $(Q_1 > 0)$ is a subset of $(Q_1 Q_3 - Q_2^2 \leq 0)$.\\\\
However, when $\Delta(\Rbb)$ is two non-nested ovals, since $(Q_1 Q_3 - Q_2^2 \leq 0)$ is the union of two disjoint disks say $D_1, D_2$, but $(Q_1 > 0) = D_3^c$ for some disk $D_3$ that's contained in either $D_1$ or $D_2$. But this means that $Y_{\Tilde{\Delta}/\Delta}(\Rbb) = D_1 \cup D_2 \cup D_3^c = \Pbb^2_\Rbb$, so we have a contradiction.
\end{proof}

\begin{proposition}
       If $Y_{\Tilde{\Delta}/\Delta}(\Rbb)$ has two connected components and the hyper-elliptic curve $\Gamma$ has four real Weierstrass points, then $\Delta(\Rbb)$ is two non-nested ovals. 
\end{proposition}

{\bf educated guess:}
\begin{proof}
It suffices for us to show that $\Delta(\Rbb)$ is not two nested ovals, so we will assume for the sake of contradiction that it is. Since $\Gamma$ has four real Weierstrass points and $Y_{\Tilde{\Delta}/\Delta}(\Rbb)$ has two connected components, the signatures of $\pi_1(\Rbb)$ must follow $(0, 4), (1, 3), (0, 4), (1, 3)$.\\\\
We want to show to show that disconnected two nested ovals implies the existence of a fiber with signature $(2, 2)$. I guess maybe the idea is that we can modify the $Q_1, Q_2, Q_3$ up to $PGL_2$-action such that $(Q_1 \geq 0)$ becomes the compliment of a disk (or equivalently $Q_1$'s associated matrix $M_1$ has signature $(2, 1)$). The intuition for this guess is that as long as $(Q_1 \geq 0) \subset (Q_1 Q_3 - Q_2^2 \leq 0)$, we could try to modify it to whatever we want?
\end{proof}

\begin{question}
Let $Q_1, Q_2, Q_3$ be the associated quadratic forms of $Y_{\Tilde{\Delta}/\Delta}(\Rbb)$, let $Q_4$ be a quadratic form such that $(Q_4 \geq 0) \cup (Q_1 Q_3 - Q_2^2 \leq 0) = (Q_1 \geq 0) \cup (Q_1 Q_3 - Q_2^2 \leq 0)$, does there exist a $PGL_2$-action on $Q_1, Q_2, Q_3$ such that the new $Q_1$ has the same signature as $Q_4$? (This might be too general)
\end{question}


\end{document}
